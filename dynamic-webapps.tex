\documentclass{beamer}
\usepackage{graphicx}
\usepackage[T1]{fontenc}

\title{Building Dynamic Websites}
\author{Howie Mao}

\begin{document}
\begin{frame}
	\titlepage
\end{frame}

\begin{frame}{Clients and Servers}
	\begin{itemize}
		\item The web runs on a client-server model
		\item A server is a program that listens on a network for requests
			and services the requests. A webserver is a server that sends
			webpages as its service.
		\item A client is a program that requests services from the server.
			Your web browser is an example of a client.
	\end{itemize}
\end{frame}

\begin{frame}{How to write a server?}
	\begin{itemize}
		\item Server calls attaches itself to a port and starts listening 
			for connections
		\item Each computer has many ports (literally numbers from 1 to 65535) 
			allowing internet traffic to be delivered to the correct
			server even if there are multiple servers on the same computer
		\item A client connects to a server by specifying its address and port
	\end{itemize}
\end{frame}

\begin{frame}{What happens when I go to a website}
	\begin{itemize}
		\item Web browser (client) connects to webserver (usually on port 80)
		\item Web browser sends an HTTP request, asking for a certain resource
		\item Server sends an HTTP response, consisting of a header,
			which contains metadata about the resource being sent, followed
			by the data for the resource itself (a web page) in the body.
	\end{itemize}
\end{frame}

\begin{frame}{Web Frameworks}
	\begin{itemize}
		\item For making web applications, you probably want more than just
			a socket library
		\item You use a web framework, which provides you with the utilities
			to deal with HTTP requests and other parts of a website
		\item In this talk, we will use Flask, a very simple framework for Python
	\end{itemize}
\end{frame}

\begin{frame}{Routing}
\end{frame}

\end{document}
