\documentclass{beamer}
\usepackage{graphicx}
\usepackage[T1]{fontenc}

\title{Building Dynamic Web Applications}
\author{Howie Mao}

\begin{document}
\begin{frame}
	\titlepage
\end{frame}

\begin{frame}{Clients and Servers}
	\begin{itemize}
		\item The web runs on a client-server model
		\item A server is a program that listens on a network for requests
			and services the requests. A webserver is a server that sends
			webpages as its service.
		\item A client is a program that requests services from the server.
			Your web browser is an example of a client.
	\end{itemize}
\end{frame}

\begin{frame}{How to write a server?}
	\begin{itemize}
		\item Server calls attaches itself to a port and starts listening 
			for connections
		\item Each computer has many ports (literally numbers from 1 to 65535) 
			allowing internet traffic to be delivered to the correct
			server even if there are multiple servers on the same computer
		\item A client connects to a server by specifying its address and port
	\end{itemize}
\end{frame}

\begin{frame}{What happens when I go to a website}
	\begin{itemize}
		\item Web browser (client) connects to webserver (usually on port 80)
		\item Web browser sends an HTTP request, asking for a certain resource
		\item Server sends an HTTP response, consisting of a header,
			which contains metadata about the resource being sent, followed
			by the data for the resource itself (a web page) in the body.
	\end{itemize}
\end{frame}

\begin{frame}{Web Frameworks}
	\begin{itemize}
		\item For making web applications, you probably want more than just
			a socket library
		\item You use a web framework, which provides you with the utilities
			to deal with HTTP requests and other parts of a website
		\item In this talk, we will use Flask, a very simple framework for Python
	\end{itemize}
\end{frame}

\begin{frame}{Routing URLs}
	\begin{itemize}
		\item The key function of any web framework
		\item Takes a URL and decides what code to run
		\item Can also use parts of the URL as data
	\end{itemize}
\end{frame}

\begin{frame}{Templating}
	\begin{itemize}
		\item Makes it easier to generate HTML (or anything else) using
			changing data.
		\item Key concepts are inheritance and substitution.
	\end{itemize}
\end{frame}

\begin{frame}{Databases}
	\begin{itemize}
		\item Allows for efficient storage and retrieval of user data.
		\item Many different kinds of databases.
		\item Relational, document-oriented, key-value, graph-based, etc.
		\item Choice of database depends on type of data stored
	\end{itemize}
\end{frame}

\begin{frame}{Sessions and Cookies}
	\begin{itemize}
		\item Allows users to stay "logged in".
		\item Server sends browser a cookie, a piece of data which the browser
			stores locally.
		\item When the site is revisited, the cookie is sent back to the server.
		\item This tells server the user is logged in already.
		\item Cookies can be given an expiration time, or can be told to
			expire once browser is closed.
	\end{itemize}
\end{frame}

\begin{frame}{Security}
	\begin{itemize}
		\item Encrypt cookies used to store login info
		\item Login credentials and session cookies should be sent in 
			HTTPS (encrypted HTTP).
		\item Passwords should be hashed before storing in database.
		\item Always escape user input before storing in a database or
			putting in a template.
	\end{itemize}
\end{frame}

\end{document}
